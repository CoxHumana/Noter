\section{Der Speisezettel}
Denne sangen finnes i mange varianter og flere språk. \lipsum[13-15]
\section{Sigurd Lies ``Hymne til Pølsen''}
Denne sangen finnes i mange varianter og flere språk. \lipsum[16-18]
\section{Star Wars}
I first heard this arrangement a few years ago before I'd seen any of the Star Wars movies, and I have to say it's a lot more entertaining now. The music itself was recognizable even back then because everyone has heard the theme song from Star Wars, even if you haven't seen the films. The lyrics, on the other hand, are all about quotes and characters that may not be quite as familiar as the music.

In any case, I'm all about Star Wars these days (yes, I jumped on the bandwagon with everyone else after watching The Force Awakens), so I was excited to receive this submission and revisit the arrangement.

The lyrics are all Star Wars references, but the arrangement is actually a medley of several different John Williams pieces. Included are the themes to:

\begin{itemize}
\item Close Encounters of the Third Kind
\item Raiders of the Lost Ark
\item Super Man
\item E.T.
\item Jaws
\item Jurassic Park
\end{itemize}

I can never decide whether my favorite part is Jurassic Park because the spoken interjections are too funny or because it's just such a gentle, expansive melody. Or because the harmonies are great there.

Sorry to geek out, but how awesome is this whole arrangement? There's something incredible about hearing a piece that is performed entirely by the human voice and still has this much depth to it. You don't hear that everyday.

I mean, this is only for four vocal lines. Your average rock song has the same number of puzzle pieces (vocals, guitar, bass, drums), but they don't fit together the same way as four lines of vocals. I feel like the A Capella texture sounds more complex.

I guess it really is more complex, at least with regard to texture. In the rock band, the drums don't change texture because they don't have pitch. Then, the bass/guitar are typically in the background, supporting the vocals. You never get true polyphony like in the beginning of this medley, where you have four voices singing You must use the force and none of them is secondary to the others.

Plus, it's just cool to hear familiar tunes in a brand new way.

Okay wow. I had forgotten how much I love choral music. I was in an A Capella group in high school, but I haven't been part of a chorus almost two years now. I may have to change that.

\section{Svensk Trøndervals}
Denne sangen finnes i mange varianter og flere språk. \lipsum[22-24]
